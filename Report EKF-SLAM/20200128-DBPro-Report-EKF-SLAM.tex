\documentclass[11pt]{article}

\usepackage{cite}
\usepackage{amsmath,amssymb,amsfonts} 
\usepackage{algorithmic}
\usepackage{graphicx}
\usepackage{textcomp}
\usepackage{xcolor}
\PassOptionsToPackage{hyphens}{url}\usepackage{url}
\usepackage{multicol}
\def\BibTeX{{\rm B\kern-.05em{\sc i\kern-.025em b}\kern-.08em
T\kern-.1667em\lower.7ex\hbox{E}\kern-.125emX}}

\author{
  Jestram, Johannes\\
  \texttt{jestram@posteo.de}
	\and
	Livert, Benedikt\\
	\texttt{benedikt.livert@gmx.de}
	\and
	Paranskij, Mark\\
	\texttt{mark.paranskij@gmail.com}
}

\title{Report Datenbankprojekt: Distributed Streaming EKF SLAM}

\begin{document}

\maketitle
\newpage

\tableofcontents
\newpage

\section{Einleitung}\label{Einleitung}
In verschiedenen technischen Disziplinen ist es nützlich, auf Grundlage des bisherigen Verhaltens eines Systems eine Voraussage über dessen künftigen Zustand treffen zu können. Ein Beispiel dafür ist die Navigation: Mithilfe der bisherigen Bewegungsrichtung und der aktuellen Geschwindigkeit lässt sich die Position des Objektes zu einem zukünftigen Zeitpunkt schätzen. Um eine möglichst verlässliche Vorhersage treffen zu können, müssen verschiedene Aspekte des betreffenden Systems gemessen werden. Da Messungen jedoch stets - in unterschiedlichem Maß - fehlerbehaftet sind, sollten diese Messfehler auch in die Betrachtung des Systemzustandes einbezogen werden. Weiterhin unterscheidet sich die Komplexität der Methoden zur Vorhersage des zukünftigen Zustandes je nach betrachtetem System. Möglicherweise wird der Zustandsübergang des Systems durch komplexe Funktionen mit vielen Parametern dargestellt. Zuletzt erscheint es auch als sinnvoll, die getroffene Vorhersage anhand neuer Messungen zu überprüfen und gegebenenfalls anzupassen. Eine Methode, um solche Zustandsübergänge vorherzusagen und zu überprüfen, ist der sogenannte Kalman Filter \cite{kalman1960new}.

Ein weiteres, vor allem durch die moderne Robotik geprägtes, Problem ist das Kartographieren einer Umgebung. Zum einen muss der autonome Roboter stets seine eigene Position innerhalb der unbekannten Umgebung kennen, zum anderen soll er eben diese Umgebung erkunden und aufzeichnen. Die Erkundung der Umgebung basiert in der Regel auf der Messung mit eingebauten Sensoren, daher ist die Positionsbestimmung der Erkannten Objekte stets relativ zur eigenen Position. Andererseits verortet sich der messende Roboter innerhalb der soeben kartierten Umgebung. Die Genauigkeit der eigenen Positionsbestimmung und der erkundeten Objekte in der Umgebung hängen also in beide Richtungen unmittelbar zusammen. Beide Messungen unterliegen jedoch Fehlern, möglicherweise gibt es sogar Aussetzer bei der Erfassung der Sensordaten. Die Problemstellung erinnert an ein Henne-Ei Problem. Diese Problemstellung wird im englischen als ‘Simultaneous Localization and Mapping’ (SLAM) bezeichnet.

Es ist möglich, den Kalman Filter zur Lösung des  SLAM Problems zu nutzen \cite{freiburg_SLAM}. Wie im Folgenden beschrieben werden wird, ist die Berechnung der Vorhersagen und deren Vergleiche mit den Messungen ein rechenaufwendiger Algorithmus.

\subsection{Motivation}
In diesem Projekt wird als Datengrundlage der Victoria Park Datensatz betrachtet. Diesem Datensatz liegt die Bewegung eines Fahrzeugs durch besagten Park zugrunde. Der Zustand des Fahrzeugs wird durch eingebaute Sensoren sowie GPS regelmäßig gemessen. Darüber hinaus ermittelt das Fahrzeug mittels eines Lasersensors vor ihm liegende Objekte.

Basierend auf diesen Messungen soll der EKF/SLAM Algorithmus implementiert werden. Da die reine Implementierung des Verfahrens basierend auf dem gegebenen Datensatz bereits erfolgt ist \cite{ute_SLAM}, wird es im Rahmen dieses Projektes darum gehen, wie ein solche Algorithmus parallelisiert werden kann. Um das EKF-SLAM Verfahren skalierbar zu machen, ist es folglich interessant dafür notwendigen Berechnungen zu parallelisieren. Es gilt zum Einen darum zu analysieren, inwiefern die Berechnungen selbst parallelisierbar sind. Weiterhin sind in den Anwendungsgebieten des Kalman Filter auch Szenarien denkbar, in denen Daten von mehr als einem System (z.B. Fahrzeug) verarbeitet werden sollen \cite{vessel}. Für solche Szenarien stellt sich die Frage, ob  und wie Daten von mehreren Fahrzeugen möglichst effektiv verarbeitet werden können.

Dieses Projekt zielt folglich darauf ab, die parallele Implementierung des EKF SLAM Algorithmus umzusetzen und zu evaluieren. Dabei wird die Software für parallele und verteilte Verarbeitung von Datenströmen Apache Flink\footnote{\url{https://flink.apache.org/}} eingesetzt.

\subsection{Extended Kalman Filter}
Der Kalman Filter wurde im Jahr 1960 von Rudolph E. Kalman im Rahmen seines Papers “A New Approach to Linear Filtering and Prediction Problems” \cite{kalman1960new} vorgestellt. Er ist ein Ansatz um Zustände zu schätzen. Der Kalman Filter minimiert die geschätzte Fehlervarianz. Der grundlegende Algorithmus funktioniert wie folgt: Zunächst wird im sog. prediction-step der Zustand eines Systems zu einem Zeitpunkt Xt auf Grundlage vorhergegangener Zustandsmessungen und einer Zustandsübergangsfunktion geschätzt. Diese Schätzung wird dann im update-step (auch correction-step) nach dem Erhalt neuer Messdaten korrigiert und der Algorithmus beginnt von vorne.

Eine attraktive Eigenschaft dieses Algorithmus ist seine rekursive Natur; zur Berechnung des geschätzten neuen Zustands sind nur die Daten der vorherigen Messung nötig, da sich  in dieser alle vorherigen Messungen zusammengefasst wiederfinden. Der Extended Kalman FIlter unterscheidet sich vom Diskreten Kalman Filter darin, dass die Zustandsübergänge durch nichtlineare Funktionen abgebildet werden. Die Nichtlinearität verkompliziert den Algorithmus, wie im folgenden gezeigt wird.

\subsubsection{Prediction-Step}
Der Zustand eines Systems kann über beliebig viele Parameter dargestellt werden. Um einen Zustand Xt in einen Zustand Xt+1 zu überführen, muss für jeden Parameter eine Funktion definiert werden, die diesen Übergang abbildet. Im falle eines Fahrzeuges erfolgt dies über die Erstellung eines Bewegungsmodells. Auf Grundlage der Sensorik und vorhandenen Messwerte muss das Bewegungsmodells den Zustand, korrekt und dargestellt in die gewünschten Metriken, berechnen.

Doch das Bewegungsmodell ist nicht der einzige Einflussfaktor für den Zustandsübergang: Messfehler und äußere Einflussfaktoren sorgen für Ungenauigkeiten. Diese Ungenauigkeiten werden im Extended Kalman Filter berücksichtigt, wie im folgenden erläutert wird.

\subsubsection{Update-Step}
Der im Prediciton-Step geschätzte Zustand wird nun basierend auf neuen, externen Beobachtungen des Zustandes korrigiert. Jedoch ist es nicht so, dass die Schätzung einfach auf die äußere Beobachtung gesetzt wird. Die Korrektur wird gedämpft um den sog. Kalman Gain. Dieser berechnet sich sowohl aus Messunsicherheiten, als auch eine Kovarianzmatrix, die die Abhängigkeiten zwischen den einzelnen Parametern des Zustands abbildet. WARUM LINEARISIERUNG Weiterhin muss das nichtlineare 

\subsection{Simultaneous Localization and Mapping}

\section{Methodik}\label{Methodik}
\subsection{Verwendete Software}
Im folgenden werden zunächst die dem Projekt zugrunde liegenden Konzepte detailliert vorgestellt. Danach wird erklärt, welche softwareseitigen Werkzeuge zur Umsetzung benutzt werden und warum. An dieser Stelle wird der Fokus auf den für das Projekt relevanten Funktionalitäten der jeweiligen Software liegen. Anschließend werden wir ausführlich auf die bereitgestellten Daten eingehen und deren Spezifika erläutern, da dieses Wissen Grundlage der Implementierung des Algorithmus ist. Als nächstes werden wir ausgewählte Designentscheidungen der Implementierung vorstellen. Die Ergebnisse bzw. die Ausgabe unserer Implementierung werden im Anschluss dokumentiert, analysiert und bewertet.

\subsubsection{Apache Flink}
Flink ist eine Software zur verteilten und parallelen Verarbeitung von Stream- oder Batchdaten. Es ist aus dem Stratosphere Projekt\footnote{\url{http://stratosphere.eu}} \cite{alexandrov_stratosphere_2014} der TU-Berlin hervorgegangen und ein Top-Level-Projekt der Apache Foundation \footnote{\url{https://apache.org}}. Die Software ist per Open Source Lizenz verfügbar und aufgrund der hohen Performanz \cite{alexandrov_stratosphere_2014} ein beliebtes Werkzeug für verteilte und skalierbare Datenverarbeitung.

Flink basiert auf der Java Virtual Machine (JVM) und ist in Scala und Java programmiert. Der Funktionsumfang, wie in Abbildung XXXXXXXXXXXXXXXXXX dargestellt umfasst verschiedene Application Programming Interfaces (APIs). 

Da das Ziel dieses Projektes die Anwendung von EKF-SLAM auf mehrere Fahrzeuge in Echtzeit ist, wird vor allem die DataStream API genutzt. Diese liefert vorgefertigte Funktionalitäten für viele übliche Aufgaben in der Verarbeitung von Datenströmen. Diese umfassen unter anderem die automatisierte parallelisierung der Verarbeitung von Datenströmen. Weiterhin ist es möglich, eingehende Daten mit sog. Schlüsseln (engl. keys) zu versehen und die weitere Verarbeitung der Daten im Folgenden abhängig von dem zugeordneten Key zu mache. Dies erfolgt über die Klassen KeyedDataStream und KeyedDataPoint. Ein KeyedDataStream verarbeitet generische Datentupel, welche jeweils mit einem Zeitstempel und einem Schlüssel versehen sind. Dabei liefern KeyedDataPoint eine Art Blaupause für den Inhalt eines einzelnen Datentupels.

Mit Hilfe solcher geschlüsselter Datenströme ist es möglich, für jeden Schlüssel einen eigenen Zustand (engl. state) zu definieren. Folglich ist es mit Schlüsseln und Status möglich, Daten und darauf basierende Berechnungen nach deren Quelle oder Zugehörigkeit klar voneinander zu trennen.

\subsubsection{InfluxDB}
InfluxDB\footnote{\url{https://www.influxdata.com/products/influxdb-overview}} ist eine für Zeitreihendaten optimierte Datenbank. Die Interaktion erfolgt in einer Anfragesprache ähnlich zu SQL. Es existieren APIs für verschiedene Programmiersprachen, unter anderem Java. 

Jedes zu speichernde Datentupel muss mit einem Zeitstempel versehen sein. Wie in Flink ist auch hier möglich, eingehende Daten anhand eines Schlüssels zu unterscheiden. Daten können in unterschiedliche Datenbanken gespeichert werden und innerhalb einer Datenbank nach Messreihen unterschieden werden. Einzelne Messungen werden in sog. Feldern gespeichert. Anzahl und Typ des Inhalts der einzelnen Felder können durch den Nutzer festgelegt werden. Dank der genannten Funktionen bietet InfluxDB eine leistungsfähige Lösung für die anwendungsspezifische Speicherung von Echtzeitdaten.

\subsection{Victoria Park Datensatz}
Die Inhalte dieses Datensatzes \footnote{\url{https://www.mrpt.org/Dataset_The_Victoria_Park}} wurden im Jahr 2006 im Victoria Park in Sydney gesammelt. Es wurde die Bewegung eines Fahrzeugs durch den Park aufgezeichnet und währenddessen vom Fahrzeug Objekte in der nahen Umgebung erfasst. Die Dauer der Aufzeichnungen ist rund 25 Minuten.

\subsubsection{Messungen}
Es existieren drei verschieden Quellen für Messungen. Die internen Sensordaten liefern die Odometrie. Diese setzt sich aus dem Lenkwinkel \(\alpha\), sowie der Geschwindigkeit \textit{v}, gemessen am linken Hinterrad, zusammen. Außerdem wurde regelmäßig die Position des Fahrzeuges via GPS bestimmt. Die Messung der in der Umgebung befindlichen Objekte erfolgt mittels eines Lasersensors.

Neben den rohen Messungen existieren unterschiedlich aufgearbeitete Datensätze. Diese unterscheiden sich in der Anzahl der enthaltenen Datenpunkte sowie der Formatierung der Werte. Die Odometriedaten werden stets gemeinsam im Format {Zeitstempel, Lenkwinkel, Geschwindigkeit} angegeben. Es existiert eine Datei mit rund 66.000 Messungen und eine mit rund 4.000.

Auch die GPS-Daten werden in mehreren Versionen geliefert. Generell ist festzustellen, dass die GPS-Messungen in regelmäßigen Abständen von 200ms geliefert werden. Jedoch sind teils starke zeitliche Lücken in den Messungen vorhanden, sodass über einen längeren Zeitraum keine Positionsbestimmung erfolgt. Auch enthält der Datensatz einige Ausreißer. Diese erkennt man daran, dass das Auto, falls diese Messungen korrekt wären, von einem zum nächsten Punkt eine viel zu große Distanz zurücklegen würde. Beide genannten Mängel wurden auch von den Autoren des Datensatzes erkannt und benannt.

Der Lasersensor ist über der vorderen Stoßstange des Fahrzeugs positioniert und vermisst einen 180-Grad Winkel. Dieser Blickwinkel ist wird mit 361 Messpunkten von rechts nach links (in Fahrtrichtung) abgetastet. Für jeden einzelnen Messpunkt speichert der Laser den Abstand des gemessenen Objektes. Die Messungen sind für Objekte, die sich höchstens 80m entfernt befinden, hinreichend präzise.

\subsubsection{Probleme mit den Daten}
Im Laufe des Projektes stellte sich der Umgang mit den Daten als das zeitaufwändigste Problem dar. Vor allem die vielen unterschiedlichen Versionen des selben Datensatzes sorgten für regelmäßige Probleme. Hier hat es sich bewährt, jede Version der Daten einzeln grafisch darzustellen und zu vergleichen.

Darüber hinaus empfanden wir den Datensatz als äußerst unbefriedigend dokumentiert. Das hatte zur Folge, dass wir viele Stunden darauf verwendeten, die Glaubhaftigkeit und Verwendbarkeit von verschiedenen Versionen der Daten zu testen. Es ging auch darum, die rohen Messwerte mit Hilfe verschiedener Funktionen so aufzubereiten, dass sie für den Kalman Filter verwendbar waren. Beispielsweise war es nicht offensichtlich, dass im durch die Autoren aufbereiteten Odometrie-Datensatz (Format {Zeitstempel, Inkrement in X, Inkrement in Y, Fahrtrichtung des Autos}) das Inkrement in X immer die Fahrtrichtung war. Deshalb muss für die Umrechnung auf ein Globales Koordinatensystem X- und Y-Inkrement mit der Fahrtrichtung des Autos verrechnet werden. 

Zusammengerechnet machten die genannten Probleme mit dem Datensatz circa die Hälfte der Projektarbeit aus. Selbst am Ende des Projektes kam es noch dazu, dass wir verschiedene Eingangsdaten ausprobieren mussten, um valide und funktionierende Ergebnisse zu erzielen.

\subsection{Implementierung}

\subsubsection{Simulatiaon mehrerer Fahrzeuge}
Da es in diesem Projekt auch um die Parallelisierung der Verarbeitung von Daten unterschiedlicher Fahrzeuge geht, haben wir den vorhandenen Datensatz mehrfach repliziert. So kann die Berechnung mehrerer Systeme und deren Zustände simuliert und mit Hilfe von Flink parallelisiert werden. Die replizierten Daten haben wir zur besseren Veranschaulichung verändert; beispielsweise wurde die Reihenfolge der Tupel umgekehrt oder nur Teile des vollständigen Datensatzes für unterschiedliche simulierte Fahrzeuge verwendet.

\subsubsection{Umsetzung in Apache Flink}
Apache Flink unterscheidet Arbeitsaufträge in Streaming- und Batch-Aufträge. Diese unterscheiden sich in der Art und Weise der Parallelisierung der Berechnungen: bei Streams müssen beispielsweise Aggregationen mit einem Fenster versehen werden, da man sie bei einem potentiell unendlich langen Datenstrom nicht auf die gesamten Daten ausführen kann. Bei Batch-Daten ist das im Gegensatz möglich.

Da die Nutzung des Kalman Filters vor allem vor dem Hintergrund von kontinuierlich, über einen langen Zeitraum eintreffenden Datenströmen sinnvoll ist, fiel die Entscheidung auf einen Streaming-Auftrag.

Abgesehen von der Implementierung des EKF/SLAM Algorithmus bestand die Kernherausforderung beim Implementieren in der Parallelisierung der Berechnungen. Dazu mussten zum einen die Daten der unterschiedlichen Fahrzeuge auch bei der Verarbeitung dem entsprechenden Fahrzeug zugeordnet werden. Da jedes Fahrzeug einen eigenen, distinkten Zustand hat, dürfen die eingehenden Daten nicht vermischt werden. Zur Umsetzung dieser Unterscheidung dienten uns vor allem KeyedDataStreams. Jedes Fahrzeug verfügt über eine eigene ID. Diese ID wird dann zum Schlüssel des KeyedDataStreams, sodass die interne Zuordnung der Eingangsdaten eindeutig und korrekt ist. Die Berechnungen können durch die Schlüssel also für jedes Fahrzeug parallelisiert werden.

Darüber hinaus


\section{Evaluation}

\section{Zusammenfassung}\label{Zusammenfassung}

\section{Diskussion}\label{Diskussion}

\section{Appendix}\label{Appendix}

\bibliography{DBPro}
\bibliographystyle{ieeetr}

\end{document}